\documentclass[12pt]{article}

\usepackage{trabalho-fds}
\usepackage{graphicx,url}
\usepackage[utf8]{inputenc}
\usepackage[brazil]{babel}
\usepackage[latin1]{inputenc}  

     
\sloppy

\title{Datasets de resultados financeiros de empresas de \\
capital aberto da CVM, enriquecidos com Proveniência\\  
e seguindo etiqueta de Resultados Replicáveis v1.1\\ 
do padrão ACM, através do uso de containers}

\author{Gilberto Gil F. Gomes Passos\inst{1}, Saulo A. Almeida\inst{1}, Valquire da S. de Jesus\inst{1} }

\address{Programa de Pós-Graduação em Informática (PPGI)\\
Universidade Federal do Rio de Janeiro (UFRJ)\\
Caixa Postal 68.530 -- Rio de Janeiro -- RJ -- Brasil -- 21941-590
  \email{\{nedel,flavio\}@inf.ufrgs.br, sauloandrade@gmail.com,
  jomi@inf.furb.br}
}

\begin{document} 

\maketitle

\begin{abstract}
  This meta-paper describes the style to be used in articles and short papers
  for SBC conferences. For papers in English, you should add just an abstract
  while for the papers in Portuguese, we also ask for an abstract in
  Portuguese (``resumo''). In both cases, abstracts should not have more than
  10 lines and must be in the first page of the paper.
\end{abstract}
     
\begin{resumo} 
  Este meta-artigo descreve o estilo a ser usado na confecção de artigos e
  resumos de artigos para publicação nos anais das conferências organizadas
  pela SBC. É solicitada a escrita de resumo e abstract apenas para os artigos
  escritos em português. Artigos em inglês deverão apresentar apenas abstract.
  Nos dois casos, o autor deve tomar cuidado para que o resumo (e o abstract)
  não ultrapassem 10 linhas cada, sendo que ambos devem estar na primeira
  página do artigo.
\end{resumo}


\section{INTRODUÇÃO}

As iniciativas de Educação Financeira, incluindo as que se voltam para jovens, tanto em espaços escolares como em ambientes não formais de ensino, têm sido defendidas e implementadas em vários países, conforme Aprea et al (2016), a reboque das ações da Organização para Cooperação do Desenvolvimento Econômico (OCDE), desde 2005. (OCDE, 2005). Neste cenário, a abordagem de contextos e noções financeiras e econômicas no currículo de Matemática da Educação Básica tem sido preconizada pelos documentos norteadores nacionais, especialmente com a recente inclusão da Educação Financeira como tema transversal e integrador na Base Nacional Comum Curricular (BNCC).

A OBInvest (Olimpíada Brasileira de Investimentos) surge no cenário nacional em agosto de 2020, como um projeto de extensão do CEFET-RJ, com o objetivo de democratizar o acesso e a reflexão acerca de temas e situações econômico-financeiras por meio de uma perspectiva de Educação Financeira para alunos do Ensino Médio de todo o Brasil. A partir de uma lente multidisciplinar e levando em consideração aspectos didáticos e metodológicos, a OBInvest busca convidar os estudantes a pensar situações e a tomar decisões, que contribuam para o desenvolvimento de habilidades e competências necessárias para a formação crítica, emancipatória e inclusiva do indivíduo, para o pleno exercício da cidadania, e também, para a possibilidade de inserção em um novo mercado de trabalho para os jovens. 

Um outro objetivo norteador da Olimpíada, é o desenvolvimento de ferramentas com o intuito de dar acesso de modo facilitado a dados importantes e fundamentais para a tomada de decisão no âmbito de finanças. Assim, tomando um dataset curado e anotado com os metadados de proveniência das demonstrações financeiras das empresas brasileiras de capital aberto, é possível promover estudantes e demais interessados em Finanças, ao estudo dos comportamentos das séries temporais dos resultados de uma empresa e assim, estabelecer uma predição dos resultados futuros. 

A ferramenta desenvolvida a partir desse dataset poderá servir para o desenvolvimento de habilidades e competências de jovens talentos interessados em Finanças e Investimentos e poderá ser explorada como uma metodologia ativa pela Olimpíada de Investimentos, preparando profissionais para aprimoramentos e certificações financeiras, bem como o enriquecimento de atividades práticas nacionais da OBInvest.
\section{TRABALHOS RELACIONADOS} \label{sec:trabalhosrelacionados}

The first page must display the paper title, the name and address of the
authors, the abstract in English and ``resumo'' in Portuguese (``resumos'' are required only for papers written in Portuguese).

The abstract and ``resumo'' (if is the case) must be in 12 point Times font,
indented 0.8cm on both sides. The word \textbf{Abstract} and \textbf{Resumo},
should be written in boldface and must precede the text.

\section{MATERIAIS E METODOLOGIAS}

In some conferences, the papers are published on CD-ROM while only the
abstract is published in the printed Proceedings. In this case, authors are
invited to prepare two final versions of the paper. One, complete, to be
published on the CD and the other, containing only the first page, with
abstract and ``resumo'' (for papers in Portuguese).

\subsection{O dado bruto}

The subsection titles must be in boldface, 12pt, flush left.

\subsection{Aquisição dos dados}

The subsection titles must be in boldface, 12pt, flush left.

\subsection{Pipeline dos dados}

The subsection titles must be in boldface, 12pt, flush left.

\subsection{Problemas encontrados nos datasets brutos}

The subsection titles must be in boldface, 12pt, flush left.

\subsection{Discussão sobre os dados, limpeza e vinculação de registros}

Section titles must be in boldface, 13pt, flush left. There should be an extra
12 pt of space before each title. Section numbering is optional. The first
paragraph of each section should not be indented, while the first lines of
subsequent paragraphs should be indented by 1.27 cm.

\section{DATASETS CURADOS, ENRIQUECIDOS COM PROVENIÊNCIA DE METADADOS, SEGUINDO ETIQUETA DE RESULTADOS REPLICÁVEIS V1.1, PADRÃO ACM}\label{sec:datasetscurados}

\begin{figure}[ht]
\centering
\includegraphics[width=.3\textwidth]{fig2.jpg}
\caption{This figure is an example of a figure caption taking more than one
  line and justified considering margins mentioned in Section~\ref{sec:figs}.}
\label{fig:exampleFig2}
\end{figure}

\subsection{Dicionário de dados}

The subsection titles must be in boldface, 12pt, flush left.

In tables, try to avoid the use of colored or shaded backgrounds, and avoid
thick, doubled, or unnecessary framing lines. When reporting empirical data,
do not use more decimal digits than warranted by their precision and
reproducibility. Table caption must be placed before the table (see Table 1)
and the font used must also be Helvetica, 10 point, boldface, with 6 points of
space before and after each caption.

\begin{table}[ht]
\centering
\caption{Variables to be considered on the evaluation of interaction
  techniques}
\label{tab:exTable1}
\includegraphics[width=.7\textwidth]{table.jpg}
\end{table}

\subsection{Datasets complementares}

The subsection titles must be in boldface, 12pt, flush left.

\subsection{Aplicando FAIR e Proviência nos Datasets Curados}

The subsection titles must be in boldface, 12pt, flush left.

Figure and table captions should be centered if less than one line
(Figure~\ref{fig:exampleFig1}), otherwise justified and indented by 0.8cm on
both margins, as shown in Figure~\ref{fig:exampleFig2}. The caption font must
be Helvetica, 10 point, boldface, with 6 points of space before and after each
caption.

\begin{figure}[ht]
\centering
\includegraphics[width=.5\textwidth]{fig1.jpg}
\caption{A typical figure}
\label{fig:exampleFig1}
\end{figure}

\subsection{Pipeline de dados FAIR}

The subsection titles must be in boldface, 12pt, flush left.

\subsection{Replicabilidade do experimento utilizando etiqueta padrão ACM}

The subsection titles must be in boldface, 12pt, flush left.


\section{DISCUSSÃO}

All images and illustrations should be in black-and-white, or gray tones,
excepting for the papers that will be electronically available (on CD-ROMs,
internet, etc.). The image resolution on paper should be about 600 dpi for
black-and-white images, and 150-300 dpi for grayscale images.  Do not include
images with excessive resolution, as they may take hours to print, without any
visible difference in the result. 

\section{CONSIDERAÇÕES FINAIS}

All images and illustrations should be in black-and-white, or gray tones,
excepting for the papers that will be electronically available (on CD-ROMs,
internet, etc.). The image resolution on paper should be about 600 dpi for
black-and-white images, and 150-300 dpi for grayscale images.  Do not include
images with excessive resolution, as they may take hours to print, without any
visible difference in the result. 

\section{DECLARAÇÕES ÉTICAS}

Bibliographic references must be unambiguous and uniform.  We recommend giving
the author names references in brackets, e.g. \cite{knuth:84},
\cite{boulic:91}, and \cite{smith:99}.

The references must be listed using 12 point font size, with 6 points of space
before each reference. The first line of each reference should not be
indented, while the subsequent should be indented by 0.5 cm.

\bibliographystyle{sbc}
\bibliography{trabalho-fds}

\end{document}
